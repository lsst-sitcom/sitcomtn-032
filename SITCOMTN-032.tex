\documentclass[SE,authoryear,toc]{lsstdoc}
% lsstdoc documentation: https://lsst-texmf.lsst.io/lsstdoc.html
\input{meta}

% Package imports go here.
\usepackage{xspace}

% Local commands go here.
\renewcommand{\eg}{\textit{e.g.}\xspace}
% Package imports go here.

% Local commands go here.

%If you want glossaries
%\input{aglossary.tex}
%\makeglossaries

\title{Visits, snaps, seqNums, and exposureIDs}

% Optional subtitle
% \setDocSubtitle{A subtitle}

\author{%
Robert Lupton
}

\setDocRef{SITCOMTN-032}
\setDocUpstreamLocation{\url{https://github.com/lsst-sitcom/sitcomtn-032}}

\date{\vcsDate}

% Optional: name of the document's curator
% \setDocCurator{The Curator of this Document}

\setDocAbstract{%
How we define visits and use them in pipeline code and notebooks is changing with RFC-836.  This note describes what this means to users
}

% Change history defined here.
% Order: oldest first.
% Fields: VERSION, DATE, DESCRIPTION, OWNER NAME.
% See LPM-51 for version number policy.
\setDocChangeRecord{%
  \addtohist{1}{2022-03-28}{Initial Version.}{Robert Lupton}
}

\begin{document}

% Create the title page.
\maketitle
% Frequently for a technote we do not want a title page  uncomment this to remove the title page and changelog.
% use \mkshorttitle to remove the extra pages

% ADD CONTENT HERE
% You can also use the \input command to include several content files.

\section{Introduction}

The LSST defines exposures by the tuple \texttt{(instr, controller, dayObs, seqNum)} (which are referred to
as \texttt{(instr, controller, day\_obs, seq\_num)} in the butler (\eg \texttt{(AT, C, 2022-03-28, 666)});
see Sec. \label{sec:imageIdentifiers} for an explanation.
When we take snaps (back-to-back exposures which will
be combined early in the DM processing), they are assigned sequential values of \texttt{seqNum}.

Rubin also has the concept of a \textit{visit}, which is used to express two distinct concepts:
\begin{itemize}
\item
  These data maps to a point on the sky (as opposed to \eg a flat field exposure)
\item
  These data were taken as snaps, and the visit is the identifier for the combined data
\end{itemize}
While logically distinct, Jim Bosch and the middleware group prefers to use visit to mean
\textit{both} of these,
as it avoids repeated ``does this visit map to a region of the sky?'' tests.

\section{Changes Proposed in (DM) \href{https://jira.lsstcorp.org/browse/RFC-836}{RFC-863}}

Most users of Rubin data will not care whether or not the data is taken using snaps, and
there are significant types of data taken on sky that are not taken as snaps; for example, intra-/extra-focal
pairs of images taken to analyze the state of the optics.

RFC-836 therefore proposes simplifying the current visit system:
\begin{itemize}
\item
  All exposures, whether consisting of snaps or not, may be referred to by their first
  \texttt{seqNum} without any need for the user to explicitly define a visit.
\item
  Visits must consist of exactly one or two snaps.
\item
  \texttt{groupId}s may not be used to define visits, but there will be selecting
  exposures identified by a \texttt{group\_id}, in much the same way that one can filter on an
  exposure type.  We expect that external tools will use this information to generate
  lists of exposures to be processed.
\end{itemize}

The restriction to one or two snaps allows us to simplify the
processing of snaps --- they are only used when the two snaps may be considered as part
of a single interrupted exposure. For \eg Deep Drilling Fields (DDFs) where we may take tens
or even hundreds of exposures, each image is assigned a separate visit.

We anticipate that \texttt{groupId}s \textit{will} be used to define groups of \eg flat field
exposures that should be handled together, or a series of exposures taken in a DDF.  If you
want to use \texttt{takeImages} with $N > 2$ this information will be stored in the butler's
exposure record, but the butler will not define a visit.

\subsection{Metadata and the EFD}

The snaps within a visit are identified by 1-indexed keywords \texttt{CURINDEX} and \texttt{MAXINDEX},
which correspond to EFD names \texttt{imageIndex} and \texttt{imagesInSequence} respectively.

\subsection{Butler Access to Snaps}

Once this work is completed (expected to be \textit{c.} mid-April 2022) users will be able to use
\texttt{(day\_obs, seq\_num)} to refer to data whether it's a single exposure or a visit
(in which case \texttt{seq\_num} will be the first of the pair of snaps, if two snaps were taken), but
this means that we will need to add special syntax to access the first of the snaps (the second is easy,
as its \texttt{seq\_num} cannot be confused with a visit identifier), \eg
\texttt{(day\_obs, seq\_num, snap)} with \texttt{snap=1,2}.  The details are not yet decided.

\section{Image identifiers}
\label{sec:imageIdentifiers}

Due to a feature in \texttt{postgres}, whereby it converts non-quoted text to lower case although
it supports case-sensitive schema, the butler maps camel-case names (`\texttt{dateObs}') to snake-case
names (`\texttt{date\_obs}').  This
document does not always follow this convention, as names in ICDs have been specified using
the former convention.  Caveat Lector.

\begin{description}
  \item[projectedVisitId]
    A unique identifier issued by the Scheduler that identifies a science visit
    (composed of one or two exposures taken with a single takeImages command)
    that is planned to be taken within N hours (N < 100).

  \item[nextVisitId]
    A unique identifier, expected to be timestamp-based, that identifies an
    exposure or visit (science or calibration) that is about to be taken within
    N seconds (N < 100).

  \item[groupId]
    A unique identifier, expected to be timestamp-based, that identifies a
    single execution (past, present, or future) of a script in the Script
    Queue.  The script may contain multiple takeImages commands.  May be used
    as the nextVisitId if there is only one takeImages command.

  \item[subGroupId]
    A unique identifier, derived from the groupId, that identifies a
    particular execution of a takeImages command in a script in the Script
    Queue.  May be used as the nextVisitId if it is available early enough.

  \item[snapId]
    Either 0 or 1.  When a two-image, on-sky, same-pointing, science visit
    is taken using the takeImages command, the first exposure is snapId 0 and
    the second is snapId 1.  No other uses of the takeImages command result in
    snaps.

  \item[dayObs]
    A natural number representing the observation day (in timezone UTC-12:00)
    when the takeImages command began executing.  Does not change during a
    takeImages command, but may change during a script.

  \item[seqNum]
    A non-negative integer representing the number of exposures taken so far
    during this observation day.  Some sequence numbers may be missing if
    exposures are aborted or cause errors/faults.  It is guaranteed that
    the seqNums associated with a single takeImages command are sequential.

  \item[exposureId]
    A unique identifier, composed of multiple fields, including the dayObs and
    seqNum.  It does not include the nextVisitId, groupId, subGroupId, or
    snapId.

  \item[firstImageId]
    A non-negative integer representing the seqNum of the first image taken by
    a given takeImages command.

  \item[lastImageId]
    A non-negative integer representing the seqNum of the last image taken (or
    to be taken) by a given takeImages command.

  \item[visitId]
    A unique identifier identical to the exposureId of a visit where the seqNum
    is equal to the firstImageId or, equivalently, the snapId is 0.
\end{description}


%------------------------------------------------------------------------------



\appendix
% Include all the relevant bib files.
% https://lsst-texmf.lsst.io/lsstdoc.html#bibliographies
\section{References} \label{sec:bib}
\renewcommand{\refname}{} % Suppress default Bibliography section
\bibliography{local,lsst,lsst-dm,refs_ads,refs,books}

% Make sure lsst-texmf/bin/generateAcronyms.py is in your path
\section{Acronyms} \label{sec:acronyms}
\addtocounter{table}{-1}
\begin{longtable}{p{0.145\textwidth}p{0.8\textwidth}}\hline
\textbf{Acronym} & \textbf{Description}  \\\hline

AT & Auxiliary Telescope \\\hline
DDF & Deep Drilling Fields \\\hline
DM & Data Management \\\hline
EFD & Engineering and Facility Database \\\hline
LSST & Legacy Survey of Space and Time (formerly Large Synoptic Survey Telescope) \\\hline
RFC & Request For Comment \\\hline
SE & System Engineering \\\hline
UTC & Coordinated Universal Time \\\hline
\end{longtable}

% If you want glossary uncomment below -- comment out the two lines above
%\printglossaries

\end{document}
